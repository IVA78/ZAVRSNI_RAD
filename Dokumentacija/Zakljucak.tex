\chapter*{Zaključak}
\addcontentsline{toc}{chapter}{Zaključak}
\noindent U završnom radu izrađena je web aplikacija "Vožnja +" koja služi  za organizaciju i kontinuirano praćenje obuke za vozača B kategorije. Glavni cilj rada bio je cjeloviti razvoj  aplikacije koji se može podijeliti na tri glavna dijela: početak u kojem je dominantno bilo upoznavanje s novim tehnologijama s naglaskom na React, sredina u kojoj se primarno radilo na dizajnu sustava i odabirom arhitekture te implementacijski dio koji predstavlja najizazovniji dio cijelog projekta.

\noindent Uspješnoj realizaciji implementacijskog dijela znatno su doprinijeli kolegiji "Baze podataka", "Objektno orijentirano programiranje", "Razvoj programske potpore za web", ali i široka programerska zajednica koja je ključna za dijeljenje znanja i resursa, ali motivacije i podrške. Najvažnije funkcionalnosti aplikacije su praćenje obuke kandidata bilježenjem svakog održanog sata, ali i kalendar koji ima mogućnost sinkronizacije između instruktora i kandidata što doprinosi boljoj organizaciji obuke, ali i praćenju radnog opterećenja instruktora. Osnovne funkcionalnosti aplikacije obogaćene su integracijom servisa Cloudinary i EmailJS koje omogućuju rad s multimedijskim sadržajem i komunikaciju elektroničkom poštom. 

\noindent Aplikacija ima potencijal za brojna unapređenja od kojih se posebno izdvaja integracija s postojećom aplikacijom za savladavanje teorijskog dijela obuke. Riječ je o aplikaciji "Autoškola" koja služi  za treniranje kandidata za polaganje ispita iz prometnih propisa. Slanje automatskih podsjetnika za termine u kalendaru putem elektroničke pošte također bi bilo značajno poboljšanje. Ako bi se sustav koristio u praksi, aplikacija ima mogućnost unapređenja sigurnosti sustava. Osim toga, postoji i mjesto za napredak u izgledu aplikacije i poboljšanju arhitekture cjelokupne aplikacije.