\chapter*{Summary}
\addcontentsline{toc}{chapter}{Summary}
\noindent This thesis consists of the "Vožnja +" web application and the accompanying documentation.\

\noindent The development of the web application for driving schools was realized through three phases: defining user requirements and acquiring competencies, designing the system and selecting the application architecture, and implementation within the Spring Boot and React frameworks. The main functionalities of the application are organizing training sessions using a calendar and tracking progress through a predefined training structure and notes entered for each session. The Cloudinary service for media content manipulation is integrated into the backend part of the application, while sending emails is enabled through the EmailJS service in the frontend part of the application. The database layer is implemented using the H2 relational database. Each step is thoroughly documented, and the final appearance of the application is presented.\

\noindent The result of the work is an efficient organizational system that facilitates instructors and driving school students in organizing and tracking the progress of driver training for category B. The application has the potential for numerous enhancements, particularly the integration of various educational content for self-learning, similar to the existing "Autoškola" application used for mastering the theoretical part of the training. Additionally, it is possible to expand its functionalities by including training for other vehicle categories. Although the application is specifically tailored for driving schools, it can be developed into a general system for organizing and tracking progress during learning.

\vspace{1cm}
\noindent Keywords: driving school, Spring Boot, React, H2, Cloudinary, EmailJS