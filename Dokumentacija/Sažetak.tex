\chapter*{Sažetak}
\addcontentsline{toc}{chapter}{Sažetak}
\noindent Ovaj završni rad sastoji se od sastoji se od web aplikacije "Vožnja +" i priložene dokumentacije. \

\noindent Izrada web aplikacije za autoškole realizirana je kroz tri faze: definiranje korisničkih zahtjeva i stjecanje kompetencija,dizajniranje sustava i odabir arhitekture aplikacije te implementacija u razvojnim okvirima Spring Boot i React. Glavne funkcionalnosti aplikacije su organiziranje obuke pomoću kalendara i praćenje napretka pomoću unaprijed definirane strukture obuke i bilješki koje se unose za održane sate obuke. Cloudinary servis za manipulaciju s medijskim sadržajem integriran je u backend dio aplikacije, dok je slanje elektroničke pošte omogućeno pomoću servisa EmailJS u frontend dijelu aplikacije. Sloj baze podataka implementiran je pomoću H2 relacijske baze podataka.Svaki korak detaljno je dokumentiran te je prikazan konačni izgled aplikacije.\

\noindent Rezultat rada je učinkovit organizacijski sustav koji olakšava instruktorima i polaznicima autoškole organizaciju i praćenje napretka obuke za vozača B kategorije. Aplikacija ima potencijal s brojna unapređenja od koji se posebno izdvaja integracija različitog edukacijskog sadržaja za samostalno učenje poput postojeće aplikacije  "Autoškola" koja služi za savladavanje teorijskog dijela obuke. Također, moguće je proširiti funkcionalnosti dodavanjem obuke za ostale kategorije vozila. Iako je aplikacija usko specijalizirana na autoškole, moguće ju je razviti u općeniti sustav za organizaciju i praćenje napretka tijekom učenja.

\vspace{1cm}
\noindent Ključne riječi: autoškola, Spring Boot, React, H2, Cloudinary, EmailJS