\chapter*{Uvod}
\addcontentsline{toc}{chapter}{Uvod}
\noindent Polaganje vozačkog ispita dio je života gotovo svakog pojedinca. Prema nedavnim istraživanjima o zemljama u kojima je najteže i najskuplje naučiti voziti, Hrvatska se našla na vrhu rang-ljestvice. Procjenjuje se da je iznos koji vozači u Hrvatskoj moraju potrošiti kako bi položili sve ispite i dobili vozačku dozvolu oko 1085 eura. Kao razlog teškog polaganja vozačkog ispita se navode strogi uvjeti koji zahtijevaju veliku količinu učenja i praćenja napretka kandidata. Rješenje problema prijenosa znanja, organizacije izvođenja obuke i kontinuiranog praćenje svakog kandidata zahtjeva određenu razinu digitalizacije i implementaciju odgovarajuće aplikacije.\\

\noindent Cilj ovog završnog rada je kreiranje aplikacije "Vožnja +" čija je osnovna zadaća unapređenje vozačke obuke olakšavajući instruktorima i polaznicima autoškola  organizaciju i praćenje usvajanja vozačkih vještina. Kako bi korisnici dobili osnovne informacije i kontakt s autoškolom, anonimnim korisnicima omogućen je pristup stranicama s osnovnim informacijama i pristup formi za slanje upita. Svaki polaznik na raspolaganju ima vlastiti kalendar, korisnički profil i stranicu s napretkom koja sadrži bilješke za svaki sat koji je održan. Instruktori imaju pregled svih kandidata kojima mogu zakazivati termine ovisno o njihovoj raspoloživosti, a nakon održanog termina instruktor unosi bilješke za održani sat. Kalendari kandidata i instruktora automatski se sinkroniziraju te se tako može učinkovito organizirati i pratiti radno vrijeme instruktora. Aplikacija administratoru pruža mogućnost registracije novih korisnika te pregled svih korisnika i njihovih kalendara u sustavu. \\ 

\noindent U nastavku ovog rada slijedi detaljan opis korisničkih zahtjeva putem kojih će se definirati glavni zahtjevi i funkcionalnosti aplikacije "Vožnja +". Poglavlje o  dizajnu i arhitekturi sustava približit će čitatelju odabranu bazu podataka i implementaciju podatkovnog sloja, sloja poslovne logike i sloja korisničkog sučelja. Nakon implementacijskih detalja, bit će dan osvrt na korištene tehnologije. Za izradu web aplikacije korišteni su razvojni okviri Spring Boot i React, dok je za bazu korištena H2 baza podataka. Na kraju dolazi pregled korisničkih sučelja za sve vrste korisnika aplikacije.